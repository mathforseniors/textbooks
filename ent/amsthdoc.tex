%% filename: amsthdoc.tex
%% version: 2.20.2
%% date: 2015/03/20
%% 
%% Copyright 1999, 2004, 2010, 2015 American Mathematical Society.
%%
%% American Mathematical Society
%% Technical Support
%% Publications Technical Group
%% 201 Charles Street
%% Providence, RI 02904
%% USA
%% tel: (401) 455-4080
%%      (800) 321-4267 (USA and Canada only)
%% fax: (401) 331-3842
%% email: tech-support@ams.org
%% 
%% This work may be distributed and/or modified under the
%% conditions of the LaTeX Project Public License, either version 1.3c
%% of this license or (at your option) any later version.
%% The latest version of this license is in
%%   http://www.latex-project.org/lppl.txt
%% and version 1.3c or later is part of all distributions of LaTeX
%% version 2005/12/01 or later.
%% 
%% This work has the LPPL maintenance status `maintained'.
%% 
%% The Current Maintainer of this work is the American Mathematical
%% Society.
%%
%% ====================================================================
%%
\documentclass[a4paper, 11pt]{report}

\usepackage{amssymb} 
\usepackage{amsthm}
\usepackage{kotex}
\usepackage[T1]{fontenc}
\usepackage{url}
\usepackage[breaklinks]{hyperref}
\usepackage{graphicx}
\usepackage{multirow, multicol, makecell, booktabs}
\usepackage[font=scriptsize]{caption}
\usepackage[font=scriptsize, margin=-.3cm]{subcaption}
%\captionsetup[subfigure]{labelfont=rm}

\usepackage{setspace}

\pagestyle{myheadings}

% Use the same version number as for the public amsthm release.
%\def\instrversion{2023.05.24}
\def\instrdate{2023년 5월 24일}

\title{기초정수론 (Elementary Number Theory)}

\author{\instrdate\\\\[.5ex]
 시니어 수학교실 (Math for Seniors)\\저자: JB\\이메일: \href{mailto:mathforseniors@gmail.com}{\texttt{mathforseniors@gmail.com}})}
\date{}

\providecommand{\qq}[1]{\textquotedblleft#1\textquotedblright}
\providecommand{\mdash}{\textemdash\penalty\hyphenpenalty}

% Some item adjustments

\setlength\leftmargini{2em}
\setlength\leftmarginii{1.5em}
\renewcommand\labelitemii{$\circ$}

% Embedded \index commands are a legacy from the time when this
% documentation was part of amsldoc. Since they don't hurt anything,
% let's leave them in. Maybe they will become useful again in the
% future. [mjd,2000/06/06]

\chardef\bslchar=`\\ % p. 424, TeXbook
\newcommand{\ntt}{%
  \fontfamily\ttdefault \fontseries\mddefault \fontshape\updefault
  \selectfont
}
\DeclareRobustCommand{\cn}[1]{{\ntt\bslchar#1}}
\DeclareRobustCommand{\cls}[1]{{\ntt#1}}
\DeclareRobustCommand{\pkg}[1]{{\ntt#1}}
\DeclareRobustCommand{\opt}[1]{{\ntt#1}}
\DeclareRobustCommand{\env}[1]{{\ntt#1}}
\DeclareRobustCommand{\fn}[1]{{\ntt#1}}

\providecommand{\qedsymbol}{\leavevmode
  \hbox to.77778em{%
  \hfil\vrule
  \vbox to.675em{\hrule width.6em\vfil\hrule}%
  \vrule\hfil}}

%%  Provide a meta-command facility; provide an alternate escape
%%  character so it can be used within the verbatim environment.

\catcode`\|=0
\begingroup \catcode`\>=13 % in LaTeX2e verbatim env makes > active
\gdef\?#1>{{\normalfont$\langle$\textit{#1}$\rangle$}}
\gdef\0{\relax}
\endgroup
\def\<#1>{{\normalfont$\langle$\textit{#1}$\rangle$}}
\def\ntnote#1{{\normalfont$^{#1}$}\hangindent.5em}

\hfuzz4pt \vbadness9999 \hbadness5000
\def\latex/{{\protect\LaTeX}}

\setlength{\textwidth}{210mm}\addtolength{\textwidth}{-2in}
\setlength{\oddsidemargin}{39pt}
\setlength{\evensidemargin}{39pt}
\addtolength{\textwidth}{-2\oddsidemargin}

\setcounter{tocdepth}{2}

%%%%%%%%%%%%%%%%%%%%%%%%%%%%%%%%%%%%%%%%%%%%%%%%%%%%%%%%%%%%%%%%%%%%%%%%

\IfFileExists{mathrsfs.sty}{\usepackage{mathrsfs}}
{\IfFileExists{euscript.sty}{\usepackage[mathscr]{euscript}}
{\let\mathscr\mathcal}}

\DeclareFontEncoding{OT2}{}{} % to enable usage of cyrillic fonts
 \newcommand{\textcyr}[1]{%
   {\fontencoding{OT2}\fontfamily{wncyr}\fontseries{m}\fontshape{n}%
    \selectfont #1}}
\newcommand{\Sha}{{\mbox{\textcyr{Sh}}}}

\newcommand{\etalchar}[1]{$^{#1}$}

\newcommand{\Z}{\mathbb{Z}}
\newcommand{\legendre}[2]{\left(\frac{#1}{#2}\right)}
\newcommand{\Q}{\mathbb{Q}}
\newcommand{\A}{\mathbb{A}}
\newcommand{\C}{\mathbb{C}}
\newcommand{\R}{\mathbb{R}}
\newcommand{\PP}{\operatorname{\mathbf{Prob}}}
\newcommand{\X}{\mathscr X}
\newcommand{\Gal}{\operatorname{Gal}}
\newcommand{\rk}{\operatorname{rank\,}_{\Z}}
\newcommand{\ord}{{\operatorname{ord}}}
\newcommand{\f}{\mathfrak f_{\chi}} 
\newcommand{\p}{\mathfrak p} 
\newcommand{\ff}{\mathfrak f} 
\newcommand{\Spec}{\operatorname{Spec}}
\newcommand{\Stab}{\operatorname{Stab}}
\newcommand{\End}{\operatorname{End}}
\newcommand{\Tr}{\operatorname{Tr}}
\newcommand{\Nm}{\operatorname{\bf{Nm}}}
\newcommand{\Ker}{\operatorname{Ker}}
\newcommand{\G}{\varGamma}
\newcommand{\kbar}{{\bar k}}
\newcommand{\isom}{\cong}
\newcommand{\ph}{\varphi}
\renewcommand{\<}{\langle}
\renewcommand{\>}{\rangle}
\renewcommand{\P}{\mathbb{P}}
\newcommand{\N}{\mathfrak{F}}

\newtheorem{thm}{Theorem}[section] 
\newtheorem{prop}[thm]{Proposition}
\newtheorem{lem}[thm]{Lemma}
\newtheorem{cor}[thm]{Corollary}
\newtheorem{pred}[thm]{Prediction}
%
\theoremstyle{definition}
\newtheorem{example}[thm]{Example}
\newtheorem{remark}{Remark}
%
%\numberwithin{equation}{section}

%%%%%%%%%%%%%%%%%%%%%%%%%%%%%%%%%%%%%%%%%%%%%%%%%%%%%%%%%%%%%%%%%%%%%%%%
%%% Double spaceing for the whole article

\doublespacing

%%%%%%%%%%%%%%%%%%%%%%%%%%%%%%%%%%%%%%%%%%%%%%%%%%%%%%%%%%%%%%%%%%%%%%%%

\begin{document}
\maketitle
\markboth{기초정수론}{기초정수론}

\begingroup
\small
\tableofcontents
\endgroup

%%%%%%%%%%%%%%%%%%%%%%%%%%%%%%%%%%%%%%%%%%%%%%%%%%%%%%%%%%%%%%%%%%%%%%%%

\newpage %%%%%%%%%%%%%%%%%%%%

\chapter{강의소개와 사전지식}

본 강좌는 정수론의 기초와 그를 이용한 작은 개인/그룹과제 해결을 통해
수학에 대한 관심과 소양을 얻기위해 만들어졌습니다. 더 중요한 목적은 
실생활에서의 응용외에 하나의 철학으로 삶의 의지와 의미를 찾기위함에 
있습니다.

\section{이상적인 수강생 모델}

강좌는 아래와 같은 조건에 해당하시는 수강생들이 공부하기에 최적화되어있습니다.

\begin{itemize}
  \item 은퇴를 하셨거나 여유시간이 많으나 의미있는 활동을 찾기힘드신 분 중
  지적 성취감을 느끼고 싶으신 분.
  \item 평소 수학/과학/공학에 관심이 많으신 분.
  \item 자극적이고 단편적인 뉴스, 유튜브등의 미디어에 중독되어 이를 극복하고자
  하시는 분.
  \item 논리적인 사고력을 키워 토론에서 자신의 의견을 합리적으로 표현하고자
  하시는 분.
  \item 수학을 통해 세상과 신을 이해하고 싶으신 분.
\end{itemize}

\section{주의사항}

본 강좌에 관한 주의사항을 아해와 같습니다.

\begin{itemize}
  \item 정수론의 소양과 관련이 적은 주제에 대해 지나치게 엄밀한 논리가 
  필요한 부분은 효과적인 강의를 위해 가급적 피하겠습니다.
  \item 수학용어들은 영어번역을 같이 표기하겠습니다. 우리가 쓰는 한글 
  수학용어는 가끔 혼동을 일으키곤 합니다. 일례로 ‘소수(小數)’와 ‘소수(素數)’
  는 동음이의어로 한글로만 썼을때 상당한 혼란을 가져올 수 있습니다. 그리고, 
  차후 논문이나 원서를 일으실때 미리 용어를 영어로 알아두시는것도 좋을꺼라 
  생각합니다.
  \item 마지막으로 가장 중요한 주의사항은 \textbf{최대한 본인들이 직접 문제를 해결
  하려고 노력해야합니다.} 이것은 절대적인 요소입니다. 수학은 미디어에서 보여지는 것과 
  달리 실제로는 수많은 시행착오와 오랜 시간의 고민을 통해서 답을 찾을 수 있습니다. 
  만약 참을 수 없을 정도로 오래 노력해도 답을 얻을 수 없을때 본 강사나 동료들과의 
  대화에서 힌트를 얻기를 바랍니다.
\end{itemize}

\section{강의토픽}

본 강좌에서 공부할 토픽들을 다음과 같습니다.

\begin{itemize}
  \item 수체계: 정수(Integers), 유리수(Rational numbers), 실수(Real Numbers), 
  복소수(Complex Numbers), 대수적 수(Algebraic Numbers), 
  초원수(Transcendental Numbers) 등.
  \item 연산(Operations), 모듈러 연산(Modular Arithmetic)
  \item 방정식(Equations), 함수(Functions)
\end{itemize}

\section{사전지식}


%%%%%%%%%%%%%%%%%%%%%%%%%%%%%%%%%%%%%%%%%%%%%%%%%%%%%%%%%%%%%%%%%%%%%%%%

\begin{thebibliography}{[MDF]}

\raggedright

\bibitem[AF]{AF} AMS Author FAQ,
  \url{http://www.ams.org/authors/author-faq}

\bibitem[MDF]{MDF} The \pkg{mdframed} package,
  Marco Daniel and Elke~Schubert, 2013/07/01, v1.9b,
%  \url{http://mirror.ctan.org/macros/latex/contrib/mdframed/mdframed.pdf}
  \url{http://mirror.ctan.org/macros/latex/contrib/mdframed}

\bibitem[NDS]{NDS} The \pkg{needspace} package,
  Peter Wilson, 2010/09/12, v1.3d,
  \url{http://mirror.ctan.org/macros/latex/contrib/needspace}

\bibitem[THT]{THT} \pkg{Thmtools} Users' Guide,
  Ulrich M. Schwarz, 2014/04/21 v66,
%  \url{http://mirror.ctan.org/macros/latex/exptl/thmtools/thmtools.pdf}
  \url{http://mirror.ctan.org/macros/latex/exptl/thmtools}

\end{thebibliography}

\end{document}
