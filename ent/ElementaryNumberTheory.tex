%% filename: amsthdoc.tex
%% version: 2.20.2
%% date: 2015/03/20
%% 
%% Copyright 1999, 2004, 2010, 2015 American Mathematical Society.
%%
%% American Mathematical Society
%% Technical Support
%% Publications Technical Group
%% 201 Charles Street
%% Providence, RI 02904
%% USA
%% tel: (401) 455-4080
%%      (800) 321-4267 (USA and Canada only)
%% fax: (401) 331-3842
%% email: tech-support@ams.org
%% 
%% This work may be distributed and/or modified under the
%% conditions of the LaTeX Project Public License, either version 1.3c
%% of this license or (at your option) any later version.
%% The latest version of this license is in
%%   http://www.latex-project.org/lppl.txt
%% and version 1.3c or later is part of all distributions of LaTeX
%% version 2005/12/01 or later.
%% 
%% This work has the LPPL maintenance status `maintained'.
%% 
%% The Current Maintainer of this work is the American Mathematical
%% Society.
%%
%% ====================================================================
%%
\documentclass[a4paper, 11pt]{report}

\usepackage{amssymb, amsmath, amsthm} 
\usepackage[finemath]{kotex}
%\usepackage[finemath]{dhucs-interword}
%\interhchar{3pt}
\setInterHangulSkip{0.8pt}
\usepackage[T1]{fontenc}
\usepackage{url}
\usepackage[breaklinks]{hyperref}
\usepackage{graphicx}
\usepackage{multirow, multicol, makecell, booktabs}
\usepackage[font=scriptsize]{caption}
\usepackage[font=scriptsize, margin=-.3cm]{subcaption}
%\captionsetup[subfigure]{labelfont=rm}

\usepackage{setspace}

\pagestyle{myheadings}

% Use the same version number as for the public amsthm release.
%\def\instrversion{2023.05.24}
\def\instrdate{2023년 5월 24일}

\title{\textbf{기초정수론 (Elementary Number Theory)}}

\author{\instrdate\\\\[.5ex]
 시니어 수학교실 (Math for Seniors)\\\\유튜브: \href{https://www.youtube.com/@mathforseniors}{\texttt{https://www.YouTube.com/@mathforseniors}}\\\\\\
 저자: JB\\\\이메일: \href{mailto:mathforseniors@gmail.com}{\texttt{mathforseniors@gmail.com}}}
\date{}

\providecommand{\qq}[1]{\textquotedblleft#1\textquotedblright}
\providecommand{\mdash}{\textemdash\penalty\hyphenpenalty}

% Some item adjustments

\setlength\leftmargini{2em}
\setlength\leftmarginii{1.5em}
\renewcommand\labelitemii{$\circ$}

% Embedded \index commands are a legacy from the time when this
% documentation was part of amsldoc. Since they don't hurt anything,
% let's leave them in. Maybe they will become useful again in the
% future. [mjd,2000/06/06]

\chardef\bslchar=`\\ % p. 424, TeXbook
\newcommand{\ntt}{%
  \fontfamily\ttdefault \fontseries\mddefault \fontshape\updefault
  \selectfont
}
\DeclareRobustCommand{\cn}[1]{{\ntt\bslchar#1}}
\DeclareRobustCommand{\cls}[1]{{\ntt#1}}
\DeclareRobustCommand{\pkg}[1]{{\ntt#1}}
\DeclareRobustCommand{\opt}[1]{{\ntt#1}}
\DeclareRobustCommand{\env}[1]{{\ntt#1}}
\DeclareRobustCommand{\fn}[1]{{\ntt#1}}

\providecommand{\qedsymbol}{\leavevmode
  \hbox to.77778em{%
  \hfil\vrule
  \vbox to.675em{\hrule width.6em\vfil\hrule}%
  \vrule\hfil}}

%%  Provide a meta-command facility; provide an alternate escape
%%  character so it can be used within the verbatim environment.

\catcode`\|=0
\begingroup \catcode`\>=13 % in LaTeX2e verbatim env makes > active
\gdef\?#1>{{\normalfont$\langle$\textit{#1}$\rangle$}}
\gdef\0{\relax}
\endgroup
\def\<#1>{{\normalfont$\langle$\textit{#1}$\rangle$}}
\def\ntnote#1{{\normalfont$^{#1}$}\hangindent.5em}

\hfuzz4pt \vbadness9999 \hbadness5000
\def\latex/{{\protect\LaTeX}}

\setlength{\textwidth}{210mm}\addtolength{\textwidth}{-2in}
\setlength{\oddsidemargin}{39pt}
\setlength{\evensidemargin}{39pt}
\addtolength{\textwidth}{-2\oddsidemargin}

\setcounter{tocdepth}{2}

%%%%%%%%%%%%%%%%%%%%%%%%%%%%%%%%%%%%%%%%%%%%%%%%%%%%%%%%%%%%%%%%%%%%%%%%

\IfFileExists{mathrsfs.sty}{\usepackage{mathrsfs}}
{\IfFileExists{euscript.sty}{\usepackage[mathscr]{euscript}}
{\let\mathscr\mathcal}}

\DeclareFontEncoding{OT2}{}{} % to enable usage of cyrillic fonts
 \newcommand{\textcyr}[1]{%
   {\fontencoding{OT2}\fontfamily{wncyr}\fontseries{m}\fontshape{n}%
    \selectfont #1}}
\newcommand{\Sha}{{\mbox{\textcyr{Sh}}}}

\newcommand{\etalchar}[1]{$^{#1}$}

\newcommand{\Z}{\mathbb{Z}}
\newcommand{\legendre}[2]{\left(\frac{#1}{#2}\right)}
\newcommand{\Q}{\mathbb{Q}}
\newcommand{\A}{\mathbb{A}}
\newcommand{\C}{\mathbb{C}}
\newcommand{\R}{\mathbb{R}}
\newcommand{\PP}{\operatorname{\mathbf{Prob}}}
\newcommand{\X}{\mathscr X}
\newcommand{\Gal}{\operatorname{Gal}}
\newcommand{\rk}{\operatorname{rank\,}_{\Z}}
\newcommand{\ord}{{\operatorname{ord}}}
\newcommand{\f}{\mathfrak f_{\chi}} 
\newcommand{\p}{\mathfrak p} 
\newcommand{\ff}{\mathfrak f} 
\newcommand{\Spec}{\operatorname{Spec}}
\newcommand{\Stab}{\operatorname{Stab}}
\newcommand{\End}{\operatorname{End}}
\newcommand{\Tr}{\operatorname{Tr}}
\newcommand{\Nm}{\operatorname{\bf{Nm}}}
\newcommand{\Ker}{\operatorname{Ker}}
\newcommand{\G}{\varGamma}
\newcommand{\kbar}{{\bar k}}
\newcommand{\isom}{\cong}
\newcommand{\ph}{\varphi}
\renewcommand{\<}{\langle}
\renewcommand{\>}{\rangle}
\renewcommand{\P}{\mathbb{P}}
\newcommand{\N}{\mathfrak{F}}

\newtheorem{thm}{정리}[section] 
%\newtheorem{prop}[thm]{프로포지션}
\newtheorem{lem}[thm]{부정리}
\newtheorem{cor}[thm]{따름정리}
\newtheorem{pred}[thm]{추측}
\newtheorem{dfn}[thm]{정의}
\newenvironment{definition}{\begin{dfn}}{\end{dfn}}
\newtheorem{example}[thm]{예}
\newtheorem{remark}[thm]{참고}
\newtheorem{exercise}[thm]{연습문제}

%
%\numberwithin{equation}{section}

%%%%%%%%%%%%%%%%%%%%%%%%%%%%%%%%%%%%%%%%%%%%%%%%%%%%%%%%%%%%%%%%%%%%%%%%
%%% Double spaceing for the whole article

\setlength{\parindent}{0pt}
\setlength{\parskip}{1em}
%\doublespacing
%\singlespacing
\onehalfspacing

%%%%%%%%%%%%%%%%%%%%%%%%%%%%%%%%%%%%%%%%%%%%%%%%%%%%%%%%%%%%%%%%%%%%%%%%

\begin{document}
\maketitle
\markboth{기초정수론}{기초정수론}

\begingroup
\small
\tableofcontents
\endgroup

%%%%%%%%%%%%%%%%%%%%%%%%%%%%%%%%%%%%%%%%%%%%%%%%%%%%%%%%%%%%%%%%%%%%%%%%

\newpage %%%%%%%%%%%%%%%%%%%%

\chapter{강의소개와 사전지식}

본 강좌는 정수론의 기초와 그를 이용한 작은 개인/그룹과제 해결을 통해
수학에 대한 관심과 소양을 얻기위해 만들어졌습니다. 더 중요한 목적은 
실생활에서의 응용외에 하나의 철학으로 삶의 의지와 의미를 찾기위함에 
있습니다.

\section{이상적인 수강생 모델}

강좌는 아래와 같은 조건에 해당하시는 수강생들이 공부하기에 최적화되어있습니다.

\begin{itemize}
  \item 은퇴를 하셨거나 여유시간이 많으나 의미있는 활동을 찾기힘드신 분 중
  지적 성취감을 느끼고 싶으신 분.
  \item 평소 수학/과학/공학에 관심이 많으신 분.
  \item 자극적이고 단편적인 뉴스, 유튜브등의 미디어에 중독되어 이를 극복하고자
  하시는 분.
  \item 논리적인 사고력을 키워 토론에서 자신의 의견을 합리적으로 표현하고자
  하시는 분.
  \item 수학을 통해 세상과 신을 이해하고 싶으신 분.
\end{itemize}

\section{주의사항}

본 강좌에 관한 주의사항은 아래와 같습니다.

\begin{itemize}
  \item 정수론의 소양과 관련이 적은 주제에 대해 지나치게 엄밀한 논리가 
  필요한 부분은 효과적인 강의를 위해 가급적 피하겠습니다.
  \item 수학용어들은 영어번역을 같이 표기하겠습니다. 우리가 쓰는 한글 
  수학용어는 가끔 혼동을 일으키곤 합니다. 일례로 ‘소수(小數)’와 ‘소수(素數)’
  는 동음이의어로 한글로만 썼을때 상당한 혼란을 가져올 수 있습니다. 그리고, 
  차후 논문이나 원서를 일으실때 미리 용어를 영어로 알아두시는것도 좋을꺼라 
  생각합니다.
  \item 마지막으로 가장 중요한 주의사항은 \textbf{최대한 본인들이 직접 문제를 해결
  하려고 노력해야합니다.} 이것은 절대적인 요소입니다. 수학은 미디어에서 보여지는 것과 
  달리 실제로는 수많은 시행착오와 오랜 시간의 고민을 통해서 답을 찾을 수 있습니다. 
  만약 참을 수 없을 정도로 오래 노력해도 답을 얻을 수 없을때 본 강사나 동료들과의 
  대화에서 힌트를 얻기를 바랍니다.
\end{itemize}

\section{강의 주제, 등록, 일정}

본 강좌에서 공부할 토픽들을 다음과 같습니다.

\begin{itemize}
  \item 수체계: 정수(Integers), 유리수(Rational numbers), 실수(Real Numbers), 
  복소수(Complex Numbers), 대수적 수(Algebraic Numbers), 
  초월수(Transcendental Numbers) 등.
  \item 연산(Operations), 모듈러 연산(Modular Arithmetic)
  \item 방정식(Equations), 함수(Functions)
\end{itemize}

등록한 수강생 수가 10명이상이 되면 좀더 즉각적인 소통을 위해 유튜브 온라인 강의도 계획하고 
있습니다. 등록은 강사의 이메일 \href{mailto:mathforseniors@gmail.com}{\texttt{mathforseniors@gmail.com}}
로 성명 또는 원하시는 예명(일례로 유튜브에서 사용하는 이름), 이메일, 짧은 자기소개와 
수강이유를 보내주시면 되겠습니다. 

첫 10강의는 차후 공개될 개인/그룹 프로젝트 문제들을 해결하기위해 필요한 정수론의 기본 
정의(Definition)/용어(Terminology)/정리(Lemma, Proposition. Theorem)들을 
공부하겠습니다.

수강생에게 부여될 프로젝트 문제들의 답은 정해진 날짜전에 강사의 이메일이나 저장소에
제출하시면 되겠습니다. 제출된 모든 답은 강사의 리뷰와 함께 모든 수강생들이 열람가능하도록 
하겠습니다. 수강생 각자 수학자라고 생각하고 자신의 답에 자부심을 가지게 되었으면 
합니다. 

\section{사전지식}

본 사전지식은 집합과 논리 그리고 명제에 대해 공부해보겠습니다. 우리는 정수론을 위해 힘을 아껴야하니
이 부분에 대해서 지나친 힘을 낭비하지 마시길 바랍니다. 바로 이해가 안되시다면 그 부분은 읽고 
지나가셔도 됩니다.

\subsection{집합(Sets)}

\begin{dfn}
집합(Set)은 \textbf{정의할 수 있는 서로 다른} 객체들의 모임이다. 주어진 집합 $S$안의 각 객체 $a$를 
$S$의 원소(Element)라고 부르고 $a \in S$라고 쓰고, 만약 어떤 객체 $b$가 집합 $S$에 속해
있지 않다면 $b \not\in S$라고 씁니다.
\end{dfn}

어떤 객체도 없는 집합도 정의할 수 있습니다. 마치 정수에서 $0$처럼 말이죠. 이 특별한 
집합을 공집합(Empty set)이라고 부르고 기호로는 $\{\}$ 또는 $\emptyset$을 씁니다.

집합을 수학적으로 표기할 때의 규정은 다음과 같습니다. 만약 집합 $A$는 원소 $a, b, c$를 갖는다고 
하면 $A = \{a, b, c\}$라고 씁니다. 자연수(Natural numbers)는 $1$부터 $1$씩 더해지는 원소들의 
집합이라고 정의되고 간단히 $\mathbb{N}$이라고 쓰는데 이것을 위와같이 모든 원소들을 다 나열해서 
쓰기는 불가능하죠? 이렇듯 원소들이 자명할때는 그냥 $\{1, 2, \ldots\}$라고 쓰기도 합니다.

자연수 중 모든 짝수들의 집합을 어떻게 표현하면 좋을까요? 물론 $\{2, 4, 8, \ldots\}$이라고 
쓸수있지만 $\{2a \mid a \in \mathbb{N}\}$라고 쓸수도 있습니다. 다시말해, 자연수 집합 
$\mathbb{N}$의 각 원소 $a$에 $2$를 곱해서 만들어지는 수들의 집합으로 표현할 수 있습니다.
그럼, 자연수 중 모든 홀수들의 집합은 어떻게 표현할 수 있을까요? 각자 생각해봅시다.

위에서 봤듯이, 모든 집합은 원소의 갯수를 셀수있는 집합과 그렇지 않은 집합으로 분류됩니다. 전자를 
유한집합(Finite set), 후자를 무한집합(Infinite set)이라고 부릅니다. 일례로 $\emptyset$은 
원소의 갯수가 $0$인 유한집합, 자연수 집합 $\mathbb{N}$은 무한집합입니다.

\begin{dfn}
  집합 $A$의 모든 원소가 집합 $B$의 원소일때 $A$는 $B$의 부분집합(Subset)이라고 하고, $A \subset B$
  라고 씁니다.
\end{dfn}

두 집합 $A$와 $B$가 같다라는 것을 수학적으로 어떻게 정의할 수 있을까요? 한 방법은 $A$에서 어떤 
원소 $a$를 꺼내서 $B$에서 $a$를 찾아서 제거하는 과정을 $A$안의 모든 원소에대해 거치면 결국
$B$는 공집합이된다로 할수있겠죠. 괜찮은 방법인데 좀더 생각해보면 같은 집합에 대한 정의를 간단히
부분집합으로도 할 수 있습니다. 즉 $A \subset B \text{ 그리고 } B \subset A$일때 $A = B$라고 
정의할 수도 있겠죠?

집합들 사이에는 연산도 존재합니다. 두 집합 $A$와 $B$를 이용해 어떻게 다른 집합을 만들어낼수 있을까요?
우리가 자주 쓰는 연산들은 다음과 같습니다.

\begin{itemize}
  \item 곱집합(Intersection): $A \cap B := \{x \mid x \in A \text{ 그리고 } x \in B\}$
  \item 합집합(Union): $A \cup B := \{x \mid x \in A \text{ 또는 } x \in B\}$
  \item 여집합(Difference set): $B - A := \{x \mid x \in B \text{ 그리고 } x \not\in A\}$
\end{itemize}

예를들어, $B = \{0, 1, 2\}$이고 $A = \{1, 3\}$이라면 $A \cap B = \{1\}$, 
$A \cup B = \{0, 1, 2, 3\}$, $B - A = \{0, 2\}$ 그리고 $A - B = \{3\}$입니다. 여기서 한가지
재밌는 점은 $A \cap B = B \cap A$이고 $A \cup B = B \cup A$이지만 $A -B$와 $B - A$는 항상
같지는 않다는 점이죠. 마치 정수들의 덧셈은 서로 항들을 교환가능하지만 뺄셈은 안되는것과 마찬가지로요.

다음에는 논리와 명제에 대해 알아보도록 하겠습니다.

\subsection{논리(Logic)와 명제(Proposition)}

\begin{dfn}
  명제(Proposition)란 참(True)이거나 거짓(False)이면서 동시에 참과 거짓이 
  아닌 주장(Statements/Assertions)입니다. 
\end{dfn}

여기서 주의할 점은 수학에서 명제란 동시에 참이거나 거짓이 될수는 없습니다. 일례로, 
"내가 지금하는 주장은 거짓이다."가 명제 $P$라고 합시다. $P$가 참이라면 내가 지금하는 
주장 $P$는 거짓이므로 서로 모순이겠죠? 반대로 $P$가 거짓이라면 바로 전 논리를 이용해
$P$가 참이 되는 모순이 발생합니다. 그리고, 아무 문장이나 다 명제가 되는것은 아닙니다.
예를 들어, "이리와봐.", "오늘 점심엔 무엇을 먹을까?" 등은 참 또는 거짓이라고 할 수 
없으므로 명제가 아닙니다.

"그리고(And)", "또는(Or)" 등의 기본적인 논리연산들은 우리가 살아오면서 다 경험적으로 
체득하셨을테니 정수론에서 필요한 몇가지 논리와 증명방법으로 넘어가겠습니다.

먼저, 수학에서 자주 나오는 논증(Arguments) 중에 두 명제 $P$와 $Q$사이에 
$$P \text{ 이면 } Q.$$
라는것이 있습니다. 예를 들면 명제 
$$\text{자연수 $a$에 대해 $a + 1$이 짝수라면 $a$는 홀수다.}$$
를 생각해봅시다. 수학에서는 이것을 다음과같이 표기합니다: 
$$P \implies Q.$$
위의 명제가 참인지 아닌지, 즉 참임 증명하는데 여러 방법들이 있을텐데, 
수학에서 종종쓰는 방법은 "$P \implies Q.$"를 증명하는 대신
"$Q \text{가 아니면 } P \text{가 아니라}.$"를 증명하는 것입니다. 
대우(Contra-postive)라고 불리는데요. 위의 예에서 $a$가 홀수가 아니라면 (즉 짝수라면) 
거기에 $1$을 더한 수 $a + 1$은 홀수가 되므로(즉 짝수가 아님으로), 증명을 할 수 있게 
되는 것이죠. \textbf{결국, 두 명제 "$P \implies Q$" 와 
"$P\text{가 아님} \implies Q\text{가 아니다}$"
는 서로 논리적으로 같은 말입니다}. 우리는 이럴때 다음과같이 표기합니다:
$$"P \implies Q" \iff "P\text{가 아님} \implies Q\text{가 아니다}".$$
주의사항: $A = B$ 와 $A \iff B$는 참이거나 거짓일 수는 있지만 수학에서 같은 의미를
갖고 있지는 않습니다.

또 비슷한 방법은 모순을 이끌어내어 증명하는 방법이 있는데, 유명한 예로 
"소수(Prime numbers)는 무한히 많다."가 있습니다. 차후에 배우겠지만 "소수는 유한하다"고
가정하고 어떤 모순을 이끌어내어 그 가정 "소수는 유한하다"는 거짓이므로, 
"소수는 무한하다"고 증명을 이끌어낼 수 있는 것이죠. 이 방법은 상당히 효과적이고 간단하죠.

다른 효과적인 증명방법으로 수학적 귀납법(Induction)이라는 것이 있습니다. 이 방법은 
무한집합 자연수를 논리적으로 이용하는것인데요. 간단한 적용사례로 다음과 같은 명제를 생각해
봅시다. 
$$\text{"모든 자연수 $n$에 대해 $1+2+3+\ldots+n = \frac{n(n+1)}{2}$이다.}"$$
먼저 $n = 1$일때를 증명합니다: $1 = 1(1+1)/2 = 2/2 = 1$. 다음, 어떤 자연수 $n$일때 
위의 명제가 참이라고 가정하고 (즉 $1+2+3+\ldots+n = {n(n+1)}/{2}$), $n+1$일때를 증명합니다:
$$1+2+3+\ldots+n+(n+1) = \frac{n(n+1)}{2} + n+1 = \frac{(n+1)(n+2)}{2}.$$
직관적으로 첫번째 $n = 1$일때 참임을 증명하고 그뒤에 커지는 자연수는 연쇄적으로 자동으로
다 참임이 증명되므로 모든 자연수에서 참임이 증명된다는 아이디어입니다.

마지막으로 어떤 명제가 거짓임을 증명하는 것도 자주 등장하는데요. 보통 처음으로 시도해볼 수 있는
방법은 반례(Counter-example)를 찾아보는 것입니다. 반례를 하나만 찾아도 되겠죠? 보통 수학자들은
정리를 증명하기 어렵지만 가치가 있는 주장을 추측(Conjecture)또는 가설(Hypothesis)이라고 
발표하는데요. 추측이 나오면 제일 먼저해보는것은 실제로 추측이 참인지 거짓인지 예를 많이 
찾아보는것입니다. 그 중에 추측에 반하는 반례를 찾으면 그 즉시 추측은 쓸모없게 되버리죠.

여기까지 정수론을 위한 필수적인 소양이었구요. 다음 챕터부터는 본격적으로 정수론에 대해 공부하겠습니다.


\chapter{수체계(Numbers)와 방정식(Equations)}

\section{자연수(Natural Numbers)와 연산들(Operations)}

먼저, 앞으로 $\mathbb{N}$라고 하면 집합으로서의 자연수 $\{0, 1, 2, 3, \ldots\}$ 뿐만아니라 
덧셈(Addition)과 곱셈(Multiplication)에 대한 연산을 가지는 집합으로 쓰겠습니다. 

덧셈은 문자나 숫자를 사용하는 경우 모두 '$+$'로 표기합니다. 예를들면, $1 + 2$, $a + b$.
곱셈은 '$\cdot$'을 쓰고, 종종 생략하기도 합니다. 
예를들면, $1\cdot2$, $ab$, $a\cdot a = a^2$, $2\cdot a = 2a$.

\begin{remark}
  $\mathbb{N}$에서 어떤 두 자연수 $a, b$를 더하거나 곱하면 그 결과가 자연수임을 알고있죠?
  다시말해 그 결과가 집합 $\mathbb{N}$ 밖으로 빠져나가는 경우는 없다와 같은 말이고, 우린 
  이럴때 '\textbf{$\mathbb{N}$은 덧셈과 곱셈에 대해 닫혀있다}.'라고 합니다. 영어로는
  '$\mathbb{N}$ is closed under the addition and multiplication.'.
\end{remark}

\begin{remark}
  $\mathbb{N}$에서 덧셈과 곱셈은 교환법칙(Commutativity)을 만족합니다. 수학적으로 보면
  모든 $a, b \in \mathbb{N}$에 대해
  $$
  a + b = b + a \text{ 그리고 } ab = ba.
  $$
  즉 두 자연수 사이의 연산순서는 상관없다, 다른말로 연산결과는 연산순서에 불변(Invariant)이다.
\end{remark}

\begin{remark}
  덧셈과 곱셈에서 결합법칙(Associativity)을 만족합니다: 모든 $a, b, c \in \mathbb{N}$에 대해 
  $$
  (a+b)+c = a+(b+c) \text{ 그리고 } a(bc) = (ab)c.
  $$
  수학에서 괄호 $(), \{\}, []$는 우선순위를 나타내는 특별한 기호입니다. $(a+b)+c$의 연산순서는
  $()$안의 연산($a+b$)을 먼저한 뒤 그 결과를 $c$와 더하는 것입니다. 결합법칙은 3개 이상의
  자연수 사이에서 괄호를 앞에 두개에 쓰던 뒤에 두개에 쓰던 같은 연산값을 갖는 것을 말합니다. 
\end{remark}

\begin{remark}
  마지막으로 3개 이상의 자연수 사이의 덧셈과 곱셈이 같이쓰일때 분배법칙(Distributivity)을 만족합니다.
  수학적으로 모든 $a, b, c \in \mathbb{N}$에 대해 
  $$
  (a+b)c = ac+bc \text{ 그리고 } a(b+c) = ab + ac.
  $$
  주의할 점은 일반적으로 '$(a+b)c \neq a(b+c)$'입니다. 예를들어
  $9 = 3\cdot 3 = (1+2)3 = 1\cdot 3 + 2\cdot 3 = 3 + 6 = 9$
  이지만 $(1+2)3 = 3\cdot 3 = 9 \neq 5 = 1\cdot 5= 1(2+3).$
\end{remark}

임의의 두 자연수 $a, b$사이에 '$=$'말고도 다른 관계가 있습니다.
눈치빠르신 분은 바로 알아챘겠지만, 크기를 결정하는 "$\le$" 와 "$\ge$"입니다. 정의하자면 다음과 같습니다:
$$
a \le b \iff \text{ 어떤 } c \in \mathbb{N} \text{에 대해 } b = a + c.
$$
참고로 이 관계는 연산과는 다른 개념이죠 (다른 점이 무엇일까요?).

예를 들면, '$0$은 $1$보다 작거나 같다'를 뜻하는 $0 \le 1$ 또는 '$1$은 $0$보다 크거나 같다'를 뜻하는 
$1 \ge 0$. 더불어, '작다'와 '크다"를 나타내는 $<$"와 "$>$"도 있습니다. 
또, '$a < b$'이면 '$a \le b$'이지만 '$a\neq b$'이란 뜻입니다.
\begin{remark}\label{remark:2_1_5}
  임의의 자연수 $a, b, c$에 대해 
  \begin{itemize}
    \item $a < b \iff b > a$.
    \item $a > b, a < b, a = b$ 중 하나만 참이고 나머지 둘은 거짓.
    \item $a < b$이고 $b < c$이면 $a < c$.
    \item $a \le b$이고 $b \le a$이면 $a = b$.
    \item $a < b \implies a + c < b + c$.
    \item $c \neq 0$이고 $a < b$이면 $ ac < bc$.
  \end{itemize}
\end{remark}
\begin{exercise}
  위에서 언급했듯이 자연수 $a, b, c$에서 일반적으로 $(a+b)c \neq a(b+c)$입니다. $(a+b)c = a(b+c)$를 
  만족하는 $a, b, c$를 찾아보세요. 
\end{exercise}

\section{자연수(Natural Numbers)와 방정식(Equations)}

$\mathbb{N}$의 어떤 상수(Constants), 변수(Variables)와 덧셈/곱셈으로 
"\textbf{잘}" 구성된 표현을 생각해봅시다. 여기서 상수란 어떤 고정된 수를 표현한 문자, 변수란 
고정이 되지 않은 또는 어떤 수인지 아직 모르는 수를 대표하는 문자입니다. 일례로 서울에 사는 특정하지 
않은 한 사람을 표현하고 싶을때 문자로 대치해서 변수로 쓸 수 있겠죠? 
미지수를 $x$라고 하고 $2x + 1$은 잘 구성된 표현이고, $1++=x$이건 아닙니다. 
왜냐하면, 표현 '$++=$'는 어떤 의미를 가지는지 정의되지않았기 때문이죠.

수학에서 두 표현 $A, B$가 있다고 하면 방정식은 $A = B$라고 쓰는 식(Formula)입니다. 
일례로, $2x + 1 = 13$, $5x^3 + 2x + 1 = 0$. 물론 여러개의 변수를 쓸 수 있죠. 
중요한 점은 방정식을 정의할때 변수와 고정된 상수(Constants)는 다른 개념이기때문에 
어떤 문자가 상수인지 변수를 잘 특정해야합니다. 예를들어, 변수 $x, y \in \mathbb{N}$와 
상수 $c \in \mathbb{N}$를 가지는 방정식 $2x + 3xy^2 + c = 0$.

어떤 주어진 방정식을 푼다라는 말은 왼쪽과 오른쪽 표현들이 같은 값을 갖게 하는 변수(들)의
특정값을 찾는다와 같은 말입니다. 다시말하면, 해(Solutions)를 찾는다와도 같은 말입니다.
실제로 쉬운 방정식을 하나 풀어보죠. 변수 $x \in \mathbb{N}$를 갖는
$$
3x + 1 = 7
$$
을 생각해 봅시다. 뺄셈과 나눗셈을 알고 계신 분들은 쉽게 풀 수 있겠지만 아직 정의하지 않은 
연산들이기 때문에 우리는 각 특정값들을 대입해서 방정식을 만족하는지 찾는 방법을 쓸 수 있겠죠.
$0$부터 $x$에 대입하면 즉 
\begin{itemize}
  \item $x = 0$이면 방정식은 $3\cdot 0 + 1 = 0 + 1 = 1 \neq 7$
  \item $x = 1$이면 $3\cdot 1 + 1 = 3 + 1 = 4 \neq 7$
  \item $x = 2$이면 $3\cdot 2 + 1 = 6 + 1 = 7$
\end{itemize}
벌써 방정식을 만족하는 $x$의 자연수 하나를 찾았네요.

\begin{exercise}
  자연수 $x = 2$외에 위의 방정식을 만족하는 다른 자연수가 또 있을까요? 있다면 찾아보시고,
  없다면 왜 없는지 증명할 수 있을까요?
\end{exercise}

그럼 다른 방정식도 풀어봅시다: 변수 $x \in \mathbb{N}$를 갖는
$$
3x + 1 = 5.
$$
위와 같은 방법으로 풀어보죠.
\begin{itemize}
  \item $x = 0$이면 방정식은 $3\cdot 0 + 1 = 0 + 1 = 1 \neq 5$
  \item $x = 1$이면 $3\cdot 1 + 1 = 3 + 1 = 4 \neq 5$
  \item $x = 2$이면 $3\cdot 2 + 1 = 6 + 1 = 7 \neq 5$
  \item $x = 3$이면 $3\cdot 3 + 1 = 9 + 1 = 10 \neq 5$
\end{itemize}
계속 다른 $x$값을 대입해 계속 시도해야 할까요? 왜 더 이상할 필요가 없을까요? 자연수에 대해 $x$가 커지면
$3x$도 커지고 그러면 $3x+1$도 커지게 되겠고, 그럼 $x$가 2 이상이면 즉 $x \ge 2$이면 $3x + 1 \ge 7$이므로
$5$와 같아질 수 없죠. 그러므로, 위의 방정식을 만족하는 자연수 $\mathbb{N}$에서의 $x$값 즉 해(Solution)는 없죠.
\begin{remark}
  앞으로는 임의의 수체계에서 특별한 언급이 없을 경우 $x, y, z$는 변수를 뜻하는 문자들로 사용하겠습니다. 
\end{remark}
먼저 덧셈에 대해서 특별한 자연수 $0$을 살펴봅시다. 임의의 자연수 $a$에대해 
$$
0 + a = 0 = a + 0
$$
을 만족하죠. 비슷하게 곱셈에 대해서도 $0$과 같은 역할을 하는 자연수 $1$이 있습니다. 
모든 $a \in \mathbb{N}$에 대해 다음을 만족하죠:
$$
1\cdot a = a \cdot 1 = a.
$$
참 절묘한 정의들이죠? $\mathbb{N}$의 덧셈에 대한 '$0$'과 $\mathbb{Z}_{>0}$의 
곱셈에 대한 '$1$'과 같은 수를 항등원(Identity)라고 부릅니다. 일반적인 연산에서의 정의는 다음과 같습니다.

\begin{dfn}[항등원] 집합 $S$가 어떤 연산 $*$에 대해 닫혀있다고하자. 만약 모든 원소 $a \in S$에 대해서
  $a*e = a = e*a$를 만족하는 어떤 원소 $e \in S$가 있을때, $e$를 $*$에 대한 항등원이라고 한다.
\end{dfn}

\iffalse
이제 방정식과 수체계사이의 관계에 대한 역사에 대해 알아보죠. 자연수 $\mathbb{N}$에 특별한 수 $0$이 왜 
포함됐을까요? 그건 덧셈과 관련이 있는데, 다음과 같은 집합과 방정식을 생각해봅시다. $\mathbb{Z}_{>0} = \mathbb{N} - \{0\}$이라고
정의하고 변수 $x \in \mathbb{Z}_{>0}$와 상수 $a \in \mathbb{Z}_{>0}$를 가지는 방정식
$$
x + 1 = a.
$$
만약 $a \ge 2$이면 위의 방정식을 만족하는 $x \in S$를 쉽게 찾을 수 있겠죠? 그런데, $a = 1$인 경우를 생각해보면,
$S$안에서는 방정식을 만족하는 $x$가 없죠? 만약 있다면, $x + 1 > 1$이니까요. 

우리는 이때 어떻게 할까요? 그냥 해가 없다고 하고 계속 $\mathbb{Z}_{>0}$만 가지고 있어야할까요? 
수학에서는 이럴 경우 $\mathbb{Z}_{>0}$를 확장하는 방식으로 발전해왔어요. 즉 방정식 
$x + 1 = 1$을 만족하는 $x$를 $0$이라고 정의하고 $\mathbb{N} = \mathbb{Z}_{>0} \cup \{0\}$로 쓰기 
시작한것이죠. 


더불어 확장된 집합 $\mathbb{N}$에서도  $0 + 0 = 0$이라고 정의하는데 왜 그럴까요? 일단 그렇게하면 $\mathbb{N}$은 $+$에 대해 닫혔있고, 
결합법칙을 만족하기 때문이죠: 예를 들어, 만약 $0 + 0 = 1$이라면
\begin{align*}
    (0 + 0) + 1 &= 1 + 1 = 2 \neq 1 = 0 + 1 = 0 + (0 + 1).
\end{align*}
그럼 $0$이 속한 곱셈은 어떻게 정의하면 좋을까요? 만약 $0\cdot 0 \neq 0$이라면 다음과 같이 
결합법칙과 분배법칙이 성립하지 않습니다. 일례로 $0 \cdot 0 = 1$이라면,
\begin{align*}
  (0 \cdot 0) \cdot 1 &= 1 \cdot 1 = 1 \neq 0 = 0 \cdot 1 = 0 \cdot (0 \cdot 1),\\
  (0 + 1) \cdot 0 &= 1 \cdot 0 = 0 \neq 1 = 1 + 0 = 0 \cdot 0 + 1 \cdot 0.
\end{align*}
그러므로, 모든 $a \in \mathbb{N}$에 대해 $0\cdot a = 0 = a \cdot 0$이라고 정의하면 교환, 결합, 분배법칙이
성립하는 멋진 결과가 나오죠.

비슷하게 곱셈에 대해서도 $0$과 같은 역할을 하는 자연수 $1$이 있죠. 모든 $a \in \mathbb{N}$에 대해 다음을
만족하죠:
$$
1\cdot a = a \cdot 1 = a.
$$
참 절묘한 정의들이죠? $\mathbb{N}$의 덧셈에 대한 '$0$'과 $\mathbb{Z}_{>0}$의 
곱셈에 대한 '$1$'과 같은 수를 항등원(Identity)라고 부릅니다. 일반적인 연산에서의 정의는 다음과 같습니다.

\begin{dfn}[항등원] 집합 $S$가 어떤 연산 $*$에 대해 닫혀있다고하자. 만약 모든 원소 $a \in S$에 대해서
  $a*e = a = e*a$를 만족하는 어떤 원소 $e \in S$가 있을때, $e$를 $*$에 대한 항등원이라고 한다.
\end{dfn}
\fi

\begin{example}
임의의 상수 $a \in \mathbb{N}$와 변수 $x \in \mathbb{N}$에 대해 다음의 방정식을 풀어봅시다.
$$
ax = a
$$
위 방정식의 해는 $a$의 값에 따라 달라지는데요, 먼저 $a \neq 0$이라고 하면 $x = 1$밖에 없습니다. 왜냐하면
일단 $a \cdot 1 = a$이므로 $x = 1$이고, '해가 $x = 1$외에는 없다'는 것은 아래와 같이 증명가능하죠.
만약 $1$외에 다른 자연수 해 $x$가 있다면 $x > 1$죠 (왜 $x = 0$은 될수없을까요?). 
그러면 참고~\ref{remark:2_1_5}의 마지막 성질에 의해
$ax > a$이고 따라서 두번째 성질에 의해 $ax \neq a$이므로 해가 $x = 1$외에는 없게되죠.
이제 $a = 0$이라면 방정식은 $0 \cdot x = 0$이 됩니다. 모든 $x \in \mathbb{N}$에 대해 $0 \cdot x = 0$
이므로, $a = 0$일때의 해는 모든 자연수 즉 $\mathbb{N}$가 되겠죠.
\end{example}

지금까지 우리는 자연수에서 정의된 방정식 중에는 해가 유일한 경우, 없는 경우, 무한히 많은 경우를 다 보았습니다.
이제, 곱셈에 대해 좀더 알아봅시다. 먼저 특별한 경우가 아니라면 이제부터 $1\cdot a$를 줄여서 
$a$라고 쓰고, $a$가 $n$번 곱해진 수를 $a^n$으로 쓰겠습니다: 예를들어, $2\cdot2 \cdot 2 = 2^3$.
그리고, 임의의 자연수는 다른 자연수들의 곱으로 나타낼 수 있죠? 예를 들어,
$12 = 1\cdot12 = 2\cdot6 = 3\cdot 4$ 등등. 어떤 자연수에 대해 이러한 곱셈표현에 쓰이는 자연수들을 
약수(Divisor/Factor)라고 합니다.

\begin{dfn}[자연수에 대한 약수]
  $a \in \mathbb{N}$와 $0 \neq b \in \mathbb{N}$에대해 $a = bx$를 만족하는 $x \in \mathbb{N}$가 존재하면, 
  $b$를 $a$의 약수라고 하고, $b \mid a$라고 표기한다.
\end{dfn}

\begin{remark}
  수학자 중에는 약수의 정의에서 $0 \neq b \in \mathbb{N}$조건 대신 $b \in \mathbb{N}$로 정의를 하긴합니다. 
  나중에 유리수에서 분모가 $0$이 되면 안되는 것을 미리 주지하기 위해 $0 \neq b \in \mathbb{N}$를 사용한것이고,
  나중에 조심만 한다면 두 정의다 가능합니다. 수학에서의 정의는 "좋은" 예술만큼 다양한 방법으로 가능합니다.
\end{remark}

나중에 나올 나눗셈을 미리 사용한다면, $b \mid a$는 즉 '$a$는 $b$로 나눠떨어진다'라는 말이죠. 
참고로 '$a$는 $b$로 나눠떨어지지 않는다'는 $a \nmid b$로 표기합니다. $12$에
대해 약수를 모두 찾으면, $1, 2, 3, 4, 6, 12$가 되겠죠. 다른 약수가 또 있을까요? 총 약수의 갯수가 $1$인 자연수가
있을까요? 그런 자연수가 몇개나 있을까요? 총 약수의 갯수가 $2$인 자연수들은 정수론에서 제일 중요한 수들이라 따로
정의할께요.

\begin{dfn}[소수(Prime Numbers)]
  만약 $1 \neq a \in \mathbb{N}$의 모든 약수가 $1$과 $a$뿐이라면 $a$를 소수라고 한다.
\end{dfn}

정의에서 봤듯이 $1$은 약수가 $1$뿐이지만 소수는 아닙니다. 그 이유는 나중에 알려드릴께요. $100$이하의 소수는 
다음과 같습니다.
\begin{align*}
2,& 3, 5, 7, 11, 13, 17, 19, 23, 29, 31, 37, 41,\\ 
43,& 47, 53, 59, 61, 67, 71, 73, 79, 83, 89, 97.
\end{align*}

\begin{exercise}
  소수들 중 유일한 짝수는 $2$밖에 없습니다. 왜 그럴까요? 
\end{exercise}

먼저 나중에 필요한 간단한 사실을 언급하겠습니다. 만약 $a, b \in \mathbb{N}$에 대해 $ab \neq 0$이면 $a \neq 0$이고 
$b \neq 0$입니다. 대우를 쓰면 간단히 증명되겠죠? 

약수에 관한 몇가지 성질을 알아보겠습니다. 먼저 임의의 소수 $p \nmid 1$ 왜냐하면 $p > 1$이므로 
$1 = p\cdot b$를 만족하는 자연수 $b$는 존재하지 않기때문이죠. 모든 $a \in \mathbb{N}$에
대해 $a = 1\cdot a$이므로 $1 \mid a$이고 $a \mid a$. 또한 $a, b, c \in \mathbb{N}$에 대해
\begin{equation}\label{chain_division}
a \mid b \text{ 이고 } b \mid c \implies a \mid c
\end{equation}
그리고,
\begin{equation}\label{cancellation_division}
ac \mid bc \implies a \mid b
\end{equation}
마지막으로,
\begin{equation}\label{sum_division}
a \mid b \text{ 이고 } a \mid c \implies a \mid (b+c).
\end{equation}
셋 모두 당연해 보이고, 증명도 정의만 따라가면 어렵지 않아요. 시간나실때 직접 한번 증명해보세요. 
다만 \eqref{sum_division}의 역(Converse)은 성립되지 않습니다: 역은 성립하지 않는 예를 찾아보세요. 

\begin{exercise}
  $0$의 약수는 무엇일까요? 
\end{exercise}

다음 섹션에서 자연수 집합 $\mathbb{N}$을 정수(Integers) 집합으로 연산들과 함께 확장하고 그 위에서
더 많은 성질을 알아봅시다.

\section{정수(Integers)와 방정식(Equations)}


TODO: 뺼셈, additive inverse, gcd, bezout, euclid's lemma, fundamental theorem of arithmetic


그럼 소수의 갯수는 유한할까요? 무한할까요? 유클리드(Euclid)는 $2,300$년전에 그에 대한 답과
증명을 아래와 같이 했습니다.

\begin{thm}
  모든 소수들의 집합을 $P$라고 하자. 그러면 $P$는 무한집합이다. 
\end{thm}

\begin{proof}
  (모순을 이용) $P$를 유한집합이라고 합시다. 그렇다면, $P$안에 모든 소수를 다 곱하면 어떤 유한한 
  자연수 $n$이 되겠죠? 이제 $q = n+1$이라고 합시다. 그렇다면 $q$는 $P$안에 어떤 소수로도 나눠떨어지지않죠.
  만약 어떤 소수 $p \in P$에 대해 $p \mid q$이면 $p \mid ((q-n) = 1)$. 이건 불가능하죠.
  그래서, $q$의 소수들은 결코 $P$안에없는 모순이 생깁니다. 그러므로, $P$를 무한해야만하죠.
\end{proof}

\section{유리수(Rational Numbers)와 방정식(Equations)}

\section{실수(Real Numbers)와 방정식(Equations)}

\section{복소수(Complex Numbers)} 

%\section{대수학의 기본정리(Fundamental Theorem of Algebra)}


%%%%%%%%%%%%%%%%%%%%%%%%%%%%%%%%%%%%%%%%%%%%%%%%%%%%%%%%%%%%%%%%%%%%%%%%

\begin{thebibliography}{[MDF]}

\raggedright

\bibitem[AF]{AF} AMS Author FAQ,
  \url{http://www.ams.org/authors/author-faq}

\bibitem[MDF]{MDF} The \pkg{mdframed} package,
  Marco Daniel and Elke~Schubert, 2013/07/01, v1.9b,
%  \url{http://mirror.ctan.org/macros/latex/contrib/mdframed/mdframed.pdf}
  \url{http://mirror.ctan.org/macros/latex/contrib/mdframed}

\bibitem[NDS]{NDS} The \pkg{needspace} package,
  Peter Wilson, 2010/09/12, v1.3d,
  \url{http://mirror.ctan.org/macros/latex/contrib/needspace}

\bibitem[THT]{THT} \pkg{Thmtools} Users' Guide,
  Ulrich M. Schwarz, 2014/04/21 v66,
%  \url{http://mirror.ctan.org/macros/latex/exptl/thmtools/thmtools.pdf}
  \url{http://mirror.ctan.org/macros/latex/exptl/thmtools}

\end{thebibliography}

\end{document}
